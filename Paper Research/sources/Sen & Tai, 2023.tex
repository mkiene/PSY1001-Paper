\section{Sleep Duration and Executive Function in Adults \parencite{2}}

\begin{centering}\subsection{Supporting}\end{centering}

\begin{enumerate}
    \item There is a quadratic relationship between sleep duration and executive function, suggesting that both short and long sleep durations are linked with poorer executive function \parencite[page 801, para. 2]{2}.

    \item "A study of around 480,000 individuals, aged 38–73 years, showed that 7 h of sleep per day was associated with the highest executive function performance, using a measure derived from specific computer-based tasks of attention and working memory." \parencite[page 804, para. 6, fig. 1]{2}.

    \item Sleep quality, rather than absolute duration, is crucial for maintaining executive function. Time spent in different sleep stages and sleep fragmentation correlates better with cognitive function \parencite[page 806, para. 1]{2}.

    \item Poor sleep is associated with a reduced brain volume, affecting cognition. This reduction is partially reversible with interventions like CPAP therapy for sleep apnea \parencite[page 807, para. 1]{2}.

    \item Diffusion tensor imaging studies show changes in brain connectivity after sleep deprivation, indicating the importance of sleep for maintaining functional brain connectivity \parencite[page 807, para. 2]{2}.

    \item "A single night of sleep deprivation has been shown to affect several components of executive function such as sustained attention, reaction time, and working memory, as well as other cognitive domains of consolidation of episodic and procedural memory" \parencite[page 803, para. 3]{2}.

    \item Sleep deprivation may lead to the accumulation of neurodegenerative proteins like beta-amyloid, which is linked to Alzheimer's disease \parencite[page 807, para. 3]{2}.

    \item The glymphatic system, responsible for clearing waste from the brain, is more active during sleep and may be disrupted by poor sleep \parencite[page 807, para. 4]{2}.

\end{enumerate}

\begin{centering}\subsection{Contrasting}\end{centering}

\begin{enumerate}
    \item Some studies using actigraphy and EEG found no significant associations between total sleep time and executive function, suggesting that the quality of sleep might be a more important factor \parencite[page 805, para. 4]{2}.

    \item Long sleep duration has been associated with an increased risk of dementia, but some studies did not find this link, likely because of focusing on mid-life sleep patterns \parencite[page 805, para. 2]{2}.

    \item Objective sleep measures sometimes conflict with self-reported data, highlighting potential biases in self-assessment of sleep duration \parencite[page 806, para. 1]{2}.

\end{enumerate}

\begin{centering}\subsection{Key Points}\end{centering}

\begin{enumerate}
    \item There is an optimal sleep duration for cognitive performance, with deviations in either direction leading to decreased executive function \parencite[page 804, fig. 1]{2}.

    \item Sleep quality, characterized by sleep stages and fragmentation, is more indicative of cognitive health than sleep duration alone \parencite[page 806, para. 1]{2}.

    \item Biological mechanisms linking sleep and cognition include brain volume changes, altered connectivity, protein accumulation, and disrupted glymphatic drainage \parencite[page 806, para. 8, fig. 2]{2}.

    \item Both short and long sleep durations may predict cognitive decline and dementia, emphasizing the need for optimal sleep maintenance \parencite[page 805, para. 2]{2}.

    \item Further research is needed to explore the causal relationships between sleep, cognition, with a focus on objective sleep measurements \parencite[page 808, para. 1, 3]{2}.

		\item "There is consistent evidence for an optimal duration of sleep for cognitive function which is relevant to the personal health of every ageing individual." \parencite[page 808, para. 2]{2}

\end{enumerate}
