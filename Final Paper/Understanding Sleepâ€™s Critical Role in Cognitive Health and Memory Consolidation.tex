\documentclass[stu, 12pt]{apa7}

\usepackage{mathptmx}
\usepackage{kantlipsum}
\usepackage{changepage}
\usepackage[american]{babel}
\usepackage{csquotes}
\usepackage[style=apa,sortcites=true,sorting=nyt,backend=biber]{biblatex}
\DeclareLanguageMapping{american}{american-apa}
\addbibresource{bibliography.bib}

\usepackage{etoolbox}
\makeatletter
\def\def@donotrepeattitle{}
\makeatother

\title{\textmd{Understanding Sleep’s Critical Role in Cognitive Health and Memory Consolidation}}

\begin{titlepage}
	\pagestyle{empty}	
\end{titlepage}

\author{Max Kiene}
\course{PSY1001-42}
\authorsaffiliations{Arapahoe Community College, Community College of Denver}
\professor{Dr. Casey Casler}
\duedate{August 1, 2024}

\leftheader{Kiene}

\begin{document}

\maketitle

To maintain optimum functionality across various neurocognitive tasks, the brain relies on consistent, high-quality sleep. During sleep, the brain cycles through two phases: rapid eye movement (REM) (1) and non-rapid eye movement (NREM) (2). NREM is subdivided into three stages, N1 to N3, each experiencing variations in muscle tone, brain wave patterns, and eye movements. The body cycles through these stages approximately four to six times each night, averaging ninety minutes per cycle. These stages are ``essential in memory consolidation, specifically procedural and declarative memory,'' and for ``repairing and regrowing tissues, building bone and muscle, and strengthening the immune system'' \parencite{7}. Sleep deprivation affects both REM and non-REM sleep, leading to reduced cognitive function \parencite{1}. Additionally, studies involving patients with insomnia point toward a marked deficiency in abstract thinking, psychomotor vigilance, and emotional identification \parencite{3}. The impact of sleep deprivation and inconsistent sleep cycles is statistically significant, and can in some cases predict cognitive decline and dementia \parencite{2, 3}, emphasizing the advantages that optimal sleep maintenance can yield. This paper intends to explore the negative effects of sleep deprivation and irregular sleep cycles and to outline ways in which the informed individual can minimize deficiencies and maximize proficiency in cognition and memory consolidation \parencite{6}.

Sleep deprivation affects the brain differently depending on the phase during which it is deprived. REM sleep deprivation ``appears to have a notable effect on exciting neurons,'' which is central to threat perception and threat-related stimuli reaction, whereas NREM sleep deprivation reduces the release of specific neurotransmitters, which can inhibit receptor ability to ``refresh and restore sensitivity'' \parencite{1}. \citeauthor{1} posit that the result of sleep deprivation in these stages is ``reduced cognition.'' Moreover, the duration of sleep deprivation can influence the severity of the deficiencies experienced: \citeauthor{8} show that prolonged partial sleep deprivation is more harmful when compared to a single night of total sleep deprivation, indicating the necessity for sleep consistency and quality.

\citeauthor{9} found that while sleep deprived, there is an imbalanced inhibition between task-related default mode network (DMN) and front parietal network (FPN) activity, as well as irregular increasing arousal activity which influences thalamic activity. This ``leads to irregular disturbance of the DMN activity and a reduced FPN activity during external tasks.'' Consequently, individuals are ``unable to maintain attention to specific tasks,'' since suppression of the DMN plays a vital role in allowing brain networks to exhibit successful behavior when confronted with cognitive tasks. \parencite{1}

In contrast, \cite{10} evaluated a hypothesis that ``sleep deprivation of a shorter duration (34–36 hours) adversely affects higher cortical function while effects on attention and vigilance tasks are relatively mild.'' Their experiment examined 29 sleep-deprived and 32 control subjects and found that performance on a test measuring sustained attention and higher cortical function resulted in ``no significant group performance differences in the hypothesized direction... One night of total sleep deprivation does not appear to impair performance on tasks that are designed to assess higher cortical functioning.'' Recall that this remains congruent with findings seen in \cite{8}, as the subjects were tested while undergoing short-term sleep deprivation.

Sleep duration also plays an important role in maintaining optimal sleep hygiene, and has an inverted parabolic relationship with executive function. This means that bidirectional deviation from the optimal sleep duration is linked to poorer executive function \parencite{2}. In order to find this vertex, an earlier study examined 479,420 individuals aged between 38-73, revealing that the optimal sleep duration for performance on ``computer-based tasks of attention and working memory'' was seven hours \parencite{11}. Importantly, \citeauthor{2} found that sleep quality is more crucial than absolute duration for maintaining executive function, that time spent in different sleep stages and sleep fragmentation is more closely correlated with cognitive function than total sleep time alone, and that poor sleep quality may be indicative of cognitive decline and dementia.

However, considering that some studies using actigraphy and EEG found no significant associations between total sleep time and executive function \parencite{12, 13}, sleep duration might be a less important factor than is represented in \citeauthor{2}'s study. It is also responsible to note that objective sleep measures sometimes conflict with self-reported data, which could imply biases in the self-assessment of sleep duration. Therefore, \citeauthor{2} conclude that ``there are several mechanisms'' through which poor sleep may contribute to impairing executive function, that the underlying process is ``likely to be multifactorial,'' and that future studies should consider the complexity of this phenomenon to ``better understand the causal nature between sleep and cognition.'' While their findings are subject to a group effect, there remains consistent evidence that an individualized optimal sleep duration does exist and that it is ``relevant to the personal health of every ageing individual.'' \parencite{2} 

Insomnia is a debilitating disease that affects the quality, duration, or frequency of sleep in afflicted individuals over a prolonged period of time. It is characterized by four criteria: (1) difficulty falling asleep, staying asleep, or nonrestorative sleep; (2) this difficulty persists through consistent opportunity to sleep; (3) this difficulty corresponds to adverse daytime symptoms; and (4) this difficulty occurs at least three times per week, having been an issue for at least one month \parencite{14}. 6-10\% of adults are afflicted with insomnia, the severity of which, along with their amount of REM sleep, is ``closely related to [neurocognitive deficits]'' \parencite{3}. When compared to healthy peers:
\begin{quote}

	``[Patients with insomnia] not only had a lower overall cognitive status, but had particular difficulties in abstract thinking and clock drawing. When we compared the patients’ performance to their objective sleep quality, we found that patients who fell asleep faster, spent less of the night awake, and spent longer in 'dream sleep' showed better performance on the cognitive screening test.'' \parencite{3}

\end{quote}
Comparable to symptoms of suboptimal sleep outlined by \cite{2}, the cognitive obstacles presented to patients with insomnia are ``similar to those seen in mild cognitive impairment, a pre-stage of Alzheimer’s. This supports the idea that poor sleep quality, if left untreated, can cause similar cognitive difficulties as those found in dementing diseases'' \parencite{3}. Indeed, a 22-year prospective study executed by \cite{15} found that ``verbal abstract reasoning, '' a deficiency also found in \cite{3}, had strong correlation to the development of Alzheimer's disease in the pre-clinical phase. In response to this, \citeauthor{3} observe that the deficiencies exhibited in insomnia patients ``could be indicative of an incipient neurodegenerative process.''

While inadequate sleep can result in cognitive deficiencies across multiple domains of daily life, a reversal of this bad habit can lead to more efficient memory and cognition, as sleep is ``known to facilitate the consolidation of memories learned before sleep as well as the acquisition of new memories to be learned after sleep'' and ``subjects who are allowed to sleep after learning typically perform better on a subsequent retrieval test than subjects who spend a comparable amount of time awake following learning'' \parencite{6}. \citeauthor{6} finds that memory processing during sleep can be enhanced by cueing memory reactivation during sleep via olfactory or auditory signals, stimulating specific brain oscillations, and targeting neurotransmitter systems pharmacologically. More specifically, ``Intensifying neocortical slow oscillations (the hallmark of slow wave sleep (SWS)) by electrical or auditory stimulation and modulating specific neurotransmitters such as noradrenaline and glutamate likewise facilitates memory processing during sleep.'' Pharmacologically, a study found that increasing availability of noradrenaline (norepinephrine), a neurotransmitter that plays several crucial roles in sleep architecture, leads to enhanced consolidation of procedural memories in a finger sequence tapping task \parencite{16}. 

Ethical considerations in using sleep interventions for cognitive enhancement remain crucial, since manipulations during sleep may happen without the individual's awareness, raising concerns about autonomy and informed consent. Additionally, ``the neurochemical milieu of neurotransmitters and hormones during sleep'' \parencite{6} is intricately optimized for memory consolidation, suggesting caution in pharmacological interventions. It's also valuable to note that manipulations of memory processing during sleep may have unintended side effects: ``It has been shown, for instance, that the reprocessing and integration of information during sleep can qualitatively change memories'' \parencite{6}. Despite these cautionary remarks, \citeauthor{6} emphasizes that sleep remains essential for "effective cognitive functioning" and that its deprivation can impair various cognitive abilities. Sleep intervention even has the potential for practical use: 
\begin{quote}

	"Although the effects of memory enhancement during sleep are reproducible and statistically significant, the practical significance of these effects is unclear. In humans, the estimated performance improvement in studies manipulating memory during sleep...is usually in the margins of 5–15\% compared to sleep conditions without treatment...Even taking into account that this number might eventually turn out to be smaller, considering that effect sizes are typically overestimated in the first studies of a new field of research, the true effect size might still be high enough to warrant practical applications." \parencite{6}. 

\end{quote}

Sleep plays a critical role in cognitive function and memory consolidation. The deprivation of REM and NREM sleep is shown to adversely affect a multitude of cognitive faculties, and sleep quality and consistency remains more imperative than duration alone; therefore, chronic sleep deficiencies such as insomnia are the subject of focused research, and a further understanding of these ailments and their causes may provide valuable insights into preventing the development of disorders like Alzheimer's disease. Pharmacological supplementation and memory stimulation for sleeping individuals have the potential to assist those with weaker memories, and research outlines ways in which to approach sleeping optimally. In light of these conclusions, ethical considerations must be seriously evaluated, as there remains the potential for adverse effects on mental health and violation of individual consent. Further studies should evaluate sleep-based interventions for memory enhancement and their long-term effect on mental health, and deeper research ought be conducted to understand the causal relationship between sleep hygiene and cognitive function. Overall, healthy sleep habits remain necessary for maintaining one's health, and should be prioritized by individuals looking to optimize their cognitive faculties and enhance their memory consolidation.

\printbibliography[title=\textmd{References}]

\end{document}

