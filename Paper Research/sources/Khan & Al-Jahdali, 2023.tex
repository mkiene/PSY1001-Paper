\section{The consequences of sleep deprivation on cognitive performance \parencite{1}}

\begin{centering}\subsection{Supporting}\end{centering}

\begin{enumerate}

	\item Studies$^1$ show that prolonged partial sleep deprivation (SD) is more harmful in comparison to a single night of total SD.

	\item Sleeping individuals experience three non-REM phases and one REM phase. A study$^2$ claims that REM SD "appears to have a notable effect on exciting neurons" and that NREM SD "reduces the normal release of specific neurotransmitters, which can affect the ability of the receptors to refresh and restore sensitivity." \citeauthor{1} posit that the result of SD in regard to these four phases is reduced cognition.

	\item \citeauthor{1} posit that as SD increases, homeostatic functions associated with the brain are increasingly impaired.

	\item A cited study$^5$ showed that a lack of sleep results in an increased amygdala hyperlimbic reaction, associated with negative emotional stimuli. This corresponds to a loss of medial prefontal cortex (mPFC) connectivity, and suggests a decrease in prefrontal inhibition signaling.

	\item A cited study$^7$ examining moral judgment in individuals afflicted with SD found that there were accordingly higher response latencies, as SD "impairs the ability to integrate cognition and emotion to pass moral judgment"

	\item "Sleep deprivation appears to disrupt memory consolidation in the hippocampus through long-term potentiation (LTP)." Similarly, a study$^{10}$ found that "The advancement of memory from an unstable to a more permanent form, requiring the NMDA receptor, is disrupted in SD."

	\item A cited study$^{13}$ found that sleep deprivation alters glutamatergic signaling through modifications in AMPA and NDMA receptor structure, reducing molecular signaling cascades due to an attenuated calcium influx, resulting in fewer permanent memories being consolidated in the brain (depicted in fig. 2A).

	\item A cited study$^{14}$ found that the functional deficits in plasticity and behavior is produced by a molecular disruption attributed to SD.

	\item A cited study$^{21}$ found that "A lack of sleep can result in the incapacity of the brain to process neural signals at optimal quantities, causing incoherent speech."

	\item A cited study$^{23}$ found that SD "...results in a lack of enzymes that repair brain cell damage caused by free radicals, which affects memory and speech." Moreover, another cited study$^{24}$ found that "This can be worsened for people experiencing long-term SD because of the degeneration of neurons due to "relentless brain activity."

	\item Clarify: A cited study$^{26}$ found that "equal inhibition means that there is consistent attentional performance in the brain."

	\item A cited study$^{27}$ found that "In a sleep-deprived state, there is an imbalanced inhibition between the task-related DMN and FPM activity, and inconsistent increasing arousal activity that influences the thalamic activity." \citeauthor{1} posit that this "leads to irregular disturbance of the DMN activity and a reduced FPN activity during external tasks." As a result, individuals are "unable to maintain attention to specific tasks," since "suppressing the DMN is vital to allow appropriate brain networks to achieve successful behavior towards tasks and goals."

	\item In a cited study$^{29}$ conducted on objective attention, "Visual tasks given to participants showed that the difficulty of the tasks was related to parietal cortex activation, and inactivation of the insular cortices, visual cortices, and the cingulate gyrus." This activation pattern was considerably lower in the SD participants than the group with complete sleep. \citeauthor{1} conclude that, based on a cited study$^{30}$, "These combined factors can cause impairment in the attentional networks essential for accurate attention performance and can lead to higher vulnerability to risks and accidents in routine life."

	\item A cited study$^{31}$ indicated that alertness and attention is "directly impacted by increased fatigue due to sleep loss."

	\item \citeauthor{1} posit that, in a SD state, an "unstable inhibition of task-related DMN and FPN activity as well as an inconsistent increasing arousal influencing activity in the thalamus" can cause "irregular signals of DMN activity and reduced FPN activity during tasks." This results in "a loss of attentiveness and working-memory functioning, improving with greater thalamic activity and less with reduced thalamic activity."

	\item A cited study$^{40}$ showed that "The strength reduction of synapses can explain the benefits of sleep on memory acquisition and consolidation, as energy is saved when counteracting the network effects of synaptic excitation and increased neuronal activity following wake periods."

	\item A cited study$^{45}$ concluded that "the coding of new images is compromised  after a night of slight SD, which reduces the slow wave  activity, exclusive of lessening total SD."

	\item A cited study concludes that "sleep deprivation can diminish the active process of the glymphatic system, leading to toxin build-up which can negatively affect the cognitive performance, motor functions and behavioral patterns."

	\item A study$^{51}$ measuring via a PET scan the amount of amyloid-beta in mice through standard sleep and sleep deprivation found that, in a one-night comparison, there was a "significant increase in the beta-amyloid levels in the thalamus and the hippocampus of the mice, demonstrating in vivo evidence of the effects of sleep deprivation on recognized neurodegenerative processes."

	\item \citeauthor{1} conclude that "The adverse consequences of SD are evident on overall behaviour and cognitive performance." More specifically, due to "fluctuations in the thalamic activity, synaptic renormalization, glymphatic system roles, DMN activity, amygdala activity and hippocampal activity," unequal stimulation in the brain occurs, leading to irregular brain activity. As a result, "an impairment in attentiveness, working memory, consolidation of memories, alertness, judgment, decision-making, and many other diminished cognitive performances will follow."

	\item Despite the lack of conclusive evidence regarding "the exact mechanisms and subsequent effects of SD," \citeauthor{1} conclude that "evidence provide proof that regardless of health, receiving inadequate sleep daily or for multiple days causes the body’s systems to gradually  decline." 

\end{enumerate}

\begin{centering}\subsection{Contrasting}\end{centering}

\begin{enumerate}

	\item Study$^7$ "used a debatable single assessment procedure of moral judgment, which can limit the generalizability of the results." \parencite{1}

	\item A study claimed that "short-term SD does not selectively affect prefrontal functioning. However, all the tests carried out in this study were derived from a neuropsychological battery test created for clinical purposes, mostly to examine brain damage. The tests could have a ceiling effect, not influenced by short-term SD, and the participants were all university students."

	\item While \citeauthor{1} have concluded that SD can cause the decreased synthesis of certain proteins such as RbAp48, which can result in reduced memory, study$^{20}$ "did not observe the effects of SD in the mice to be able to conclude whether sleep is the main cause of reduced memory."

	\item In contrast to studies$^{23,24}$, Cirelli et al found no apparent evidence of brain cell degeneration after long-term SD in rats. \citeauthor{1} assert that this indicates that "more research is required to determine the cause of the neuron degeneration."

	\item In contrast to study$^{31}$, Kuhnetal$^{32}$ found that glucose levels increase in SD individuals, indicating that "further research may be required to conform the effect of the glucose levels on attention and alertness in SD individuals."

	\item In contrast to study$^{45}$, Gais et al$^{11}$ "concluded through fMRI studies that declarative  memory is not affected with long term SD, and Voderholzer et a$^{46}$ that long term SD does not affect long term declarative memory in adolescents."

	\item In contrast to study$^{51}$, the PET scan technique could not differentiate between soluble and insoluble beta-amyloid. This limitation could impact the results, as soluble beta-amyloid is more predictive of neurodegenerative disorders such as Alzheimer’s disease than insoluble beta-amyloid \parencite{1}.

	\item "...more research is required to conclude whether the increase of beta amyloid is indeed due to an impared glymphatic system, specifically due to sleep loss" 

	\item \citeauthor{1} conclude that "Additional research is required to provide evidence of the validity of the exact mechanisms and subsequent effects of SD, which can be achieved with more resources, study and time."

\end{enumerate}

\begin{centering}\subsection{Key Points}\end{centering}

\begin{enumerate}

    \item Prolonged partial sleep deprivation (SD) is more detrimental to cognitive performance than a single night of total SD (1).

    \item SD affects both REM and non-REM sleep, leading to reduced cognitive function by impairing neuron excitation and neurotransmitter sensitivity (2).

    \item Sleep deprivation impairs homeostatic brain functions, emotional regulation, and moral judgment, often causing increased response latencies and impaired cognition-emotion integration (5, 7).

    \item Memory consolidation is disrupted during SD due to alterations in glutamatergic signaling and NMDA receptor function, impacting long-term memory storage (10, 13).

    \item SD leads to inconsistent brain network activity, affecting attention and working memory, due to disruptions in the DMN and FPN, and irregular thalamic signals (27, 31).

    \item Despite findings supporting the negative impact of SD on cognitive performance, several contrasting studies question the extent and mechanisms of these effects, indicating the need for further research (7, 23, 31).

    \item Conflicting evidence regarding SD's effect on declarative memory, glucose levels, and neurodegenerative processes suggests the need for more comprehensive studies (45, 46, 51).

    \item Overall, while there is substantial evidence of SD's adverse effects on cognitive and behavioral functions, further research is necessary to confirm the exact mechanisms and long-term consequences (1).
    
\end{enumerate}
