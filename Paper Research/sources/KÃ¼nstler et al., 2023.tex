\section{The Relationship Between Cognitive Impairments and Sleep Quality Measures in Persistent Insomnia Disorder \parencite{3}}

\begin{centering}\subsection{Supporting}\end{centering}

\begin{enumerate}

	\item When compared to their healthy peers, "[patients with insomnia] not only had a lower overall cognitive status, but had particular difficulties in abstract thinking and clock drawing. When we compared the patients’ performance to their objective sleep quality, we found that patients who fell asleep faster, spent less of the night awake, and spent longer in “dream sleep” showed better performance on the cognitive screening test."

	\item Interestingly, the cognitive obstacles presented to patients with insomnia are "similar to this seen in mild cognitive impairment, a pre-stage of Alzheimer’s. This supports the idea that poor sleep quality, if left untreated, can cause similar cognitive difficulties as those found in dementing diseases."

	\item "Moreover, in a 22-year prospective study$^{30}$, verbal abstract reasoning was one of the strongest predictors for the development of Alzheimer’s disease in the pre-clinical phase. Thus, this specific pattern of deficits observed here could be indicative of an incipient neurodegenerative process." 

	\item "Insomnia severity and the amount of REM-sleep are closely related to [neurocognitive deficits]" \parencite{3}

\end{enumerate}

\begin{centering}\subsection{Contrasting}\end{centering}

\begin{enumerate}

	\item In contrast to findings by \cite{1}, \cite{5} observed that "Sensorimotor speed, spatial learning and memory, working memory, abstraction and mental flexibility, emotion identification, abstract reasoning, cognitive throughput, and risk decision making were not significantly affected by sleep debt." However, \cite{4} reported different results, noting significant impairments due to sleep deprivation: fewer emotions were correctly identified, performance on the psychomotor vigilance test (PVT) was slower and less accurate, and subjects exhibited increased ratings of tiredness, sleepiness, physical exhaustion, mental fatigue, poor sleep quality, and workload during sleep restriction phases in NASA’s Human Exploration Research Analog (HERA).

	\item Whether or not REM sleep is protective against neurodegeneration is the subject of further investigation.

\end{enumerate}

\begin{centering}\subsection{Key Points}\end{centering}

\begin{enumerate}

	\item Healthy subjects unaffected by insomnia are statistically more proficient at a multitude of cognitive tasks, including abstract thinking, clock drawing, verbal abstract reasoning, emotion identification, psychomotor vigilance, and are likelier to report mental fatigue, poor sleep quality, workload, and tiredness.

\end{enumerate}


