\section{Sleep for cognitive enhancement \parencite{6}}

\begin{centering}\subsection{Supporting}\end{centering}

\begin{enumerate}

    \item "Losing even a few hours of sleep can have detrimental effects on a wide variety of cognitive processes such as attention, language, reasoning, decision making, learning and memory" \parencite{6}.

    \item "Sleep is known to facilitate the consolidation of memories learned before sleep as well as the acquisition of new memories to be learned after sleep" \parencite{6}.

    \item "Memory processing during sleep can be boosted by (i) cueing memory reactivation during sleep; (ii) stimulating sleep-specific brain oscillations; and (iii) targeting specific neurotransmitter systems pharmacologically" \parencite{6}.

    \item "Subjects who are allowed to sleep after learning typically perform better on a subsequent retrieval test than subjects who spend a comparable amount of time awake following learning" \parencite{6}.

    \item "Olfactory and auditory cues can be used, for example, to increase reactivation of associated memories during post-learning sleep" \parencite{6}.

    \item "Intensifying neocortical slow oscillations (the hallmark of slow wave sleep (SWS)) by electrical or auditory stimulation and modulating specific neurotransmitters such as noradrenaline and glutamate likewise facilitates memory processing during sleep" \parencite{6}.

\end{enumerate}

\begin{centering}\subsection{Contrasting}\end{centering}

\begin{enumerate}

	\item "The practical significance of these effects [memory enhancement during sleep] is unclear... Even taking into account that [the percentage enhancement] might eventually turn out to be smaller, considering that effect sizes are typically overestimated in the first studies of a new field of research, the true effect size might still be high enough to warrant practical applications" \parencite{6}.

	\item "Manipulating neurotransmitter systems to enhance memory during sleep has revealed inconsistent results... these findings should therefore be interpreted with caution" \parencite{6}.

  \item "Sleep-specific manipulations have been found to effectively boost cognitive functions beyond the boundaries of the normal condition... Yet, these manipulations do not create new memories, but rather enhance the stability and resistance to forgetting of existing ones" \parencite{6}.

  \item "Manipulations of memory processing during sleep can have side effects and unintended effects... It has been shown, for instance, that the reprocessing and integration of information during sleep can qualitatively change memories" \parencite{6}.

\end{enumerate}

\begin{centering}\subsection{Key Points}\end{centering}

\begin{enumerate}

	\item "Sleep is essential for effective cognitive functioning" and its deprivation can impair various cognitive abilities, but it also has the potential to enhance cognitive performance beyond normal levels \parencite{6}.

	\item Enhancing cognitive performance through sleep involves targeted manipulations such as "cueing memory reactivation during sleep," "stimulating sleep-specific brain oscillations," and "targeting specific neurotransmitter systems" \parencite{6}.

  \item Ethical considerations in using sleep for cognitive enhancement are critical, as manipulations during sleep might happen without the individual's awareness, raising concerns about autonomy and informed consent \parencite{6}.

  \item "The neurochemical milieu of neurotransmitters and hormones during sleep" is intricately optimized for memory consolidation, suggesting caution in pharmacological interventions \parencite{6}.

\end{enumerate}
