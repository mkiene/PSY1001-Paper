\section{Outline}

\subsection{Introduction}
\begin{itemize}
    \item Importance of sleep for cognitive functions and overall health.
    \item Overview of sleep stages and their roles in brain functioning.
    \item Thesis Statement: Examination of the impact of sleep deprivation and poor sleep quality on cognitive performance.
\end{itemize}

\subsection{The Consequences of Sleep Deprivation on Cognitive Performance}
\subsubsection{Findings from Khan \& Al-Jahdali (2023)}
\begin{itemize}
    \item Discussion on the effects of REM and non-REM sleep deprivation.
    \item Synthesis of studies showing the link between lack of sleep and reduced cognitive abilities including attention, memory, and decision-making.
\end{itemize}
\subsubsection{Supportive Studies}
\begin{itemize}
    \item Detailed analysis of empirical data supporting the thesis.
    \item Impact of sleep deprivation on neurobiological functions and cognitive processes.
\end{itemize}
\subsubsection{Contrasting Views}
\begin{itemize}
    \item Presentation of studies with differing viewpoints or results.
    \item Critical examination of the methodologies and conclusions of these studies.
\end{itemize}
\subsubsection{Key Points Summary}
\begin{itemize}
    \item Summary of the main findings and their implications for cognitive health.
\end{itemize}

\subsection{Sleep Duration and Executive Function in Adults (Sen \& Tai, 2023)}
\subsubsection{Core Findings and Theoretical Implications}
\begin{itemize}
    \item Examination of the relationship between sleep duration and executive functioning.
    \item Analysis of optimal sleep durations for maintaining cognitive abilities.
\end{itemize}
\subsubsection{Supportive and Contrasting Studies}
\begin{itemize}
    \item Review of supportive empirical evidence.
    \item Discussion on discrepancies and challenges in existing research.
\end{itemize}
\subsubsection{Conclusion of Section}
\begin{itemize}
    \item Concluding remarks on how sleep duration affects executive function.
\end{itemize}

\subsection{The Relationship Between Cognitive Impairments and Sleep Quality in Persistent Insomnia Disorder (Künstler et al., 2023)}
\subsubsection{Analysis of Study Findings}
\begin{itemize}
    \item Discussion on how chronic insomnia affects cognitive functions.
    \item Insights into the long-term cognitive impairments associated with poor sleep quality.
\end{itemize}

\subsection{Sleep for Cognitive Enhancement (Diekelmann, 2014)}
\subsubsection{Overview of Cognitive Benefits of Sleep}
\begin{itemize}
    \item Review of how effective sleep enhances cognitive processes.
    \item Discussion on methods to enhance cognitive functions through better sleep.
\end{itemize}

\subsection{Conclusion}
\subsection{Conclusion}
\begin{itemize}
    \item \textbf{Summary of Key Findings:} Recap the major insights derived from the study, including the critical role of sleep in cognitive function and memory consolidation. Highlight the specific impacts of REM and NREM sleep on cognitive abilities and the negative effects of sleep deprivation. Reiterate findings on the importance of sleep quality over mere duration in maintaining cognitive health.
    
    \item \textbf{Implications of Insomnia and Sleep Deprivation:} Summarize the consequences of chronic sleep deficiencies as demonstrated in insomnia patients, emphasizing the potential for long-term cognitive decline. Discuss how these findings relate to broader concerns about public health, particularly the risks of developing neurodegenerative diseases.
    
    \item \textbf{Potential of Sleep Interventions:} Briefly revisit the discussion on enhancing memory and cognitive functions through targeted sleep interventions, such as pharmacological aids and sensory cues during sleep. Consider the implications of these interventions for both therapeutic and enhancement purposes in clinical and everyday settings.
    
    \item \textbf{Ethical Considerations:} Summarize the ethical concerns mentioned, including issues related to autonomy, informed consent, and the potential unintended consequences of manipulating sleep processes.
    
    \item \textbf{Future Research Directions:} Suggest areas for further study, such as refining sleep-based interventions for memory enhancement or exploring the long-term effects of these interventions on mental health. Emphasize the need for more comprehensive studies to understand the causal relationships between sleep patterns and cognitive functions.
    
    \item \textbf{Final Thoughts:} Stress the necessity of maintaining healthy sleep habits and the potential personal and societal benefits of improving sleep hygiene. Conclude with a call to action for both individuals and policymakers to prioritize sleep health based on the findings discussed in the paper.
\end{itemize}
